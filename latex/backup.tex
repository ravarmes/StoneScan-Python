\documentclass[12pt]{article}

\usepackage{sbc-template}
\usepackage{graphicx,url}
\usepackage[utf8]{inputenc}
\usepackage[brazil]{babel}
\usepackage{xcolor}


\sloppy

\title{StoneScan: Inteligência Artificial para Reconhecimento de Padrões Visuais em Materiais Naturais}

% NÃO INSERIR NOMES DE AUTORES OU INSTITUIÇÕES
% A submissão deve ser ANÔNIMA.

\author{Autor 1\inst{1}, Autor 2\inst{1}, Autor 3\inst{1} }

\address{Instituição de Pesquisa Anônima 1 -- Departamento de Ciência da Computação\\
	Endereço Omitido por Anonimato
	 \email{autor1@gmail.com, autor2@gmail.com,
  autor3@gmail.com}
}

\begin{document}

\maketitle

\begin{abstract}
O presente trabalho propõe o desenvolvimento de um sistema baseado em \textbf{Visão Computacional} para a classificação automatizada de materiais pétreos ornamentais a partir de imagens digitais. O objetivo é validar uma metodologia de \textit{Deep Learning} que utiliza Redes Neurais Convolucionais (CNNs) para identificar e diferenciar padrões geológicos sutis, reduzindo a subjetividade inerente aos métodos de inspeção humana. A motivação primária reside na busca pela padronização e aumento da eficiência no processo de identificação de materiais em um setor de elevada relevância econômica global. A contribuição deste estudo é de natureza tecnológica e metodológica, demonstrando a viabilidade da aplicação de técnicas de \textit{transfer learning} com arquiteturas como a ResNet para problemas de classificação visual em geociências. Espera-se obter um modelo de prova de conceito com alta acurácia, que sirva como base para a transformação digital das práticas de classificação de materiais naturais.

\textcolor{red}{\textbf{Instrução CONJUNTA:} Revisar e refinar o Resumo para garantir que o número de palavras esteja estritamente abaixo do limite de 200, focando apenas nos objetivos, contribuição e resultados mais relevantes.}
\end{abstract}

\section{Introdução}
O avanço das tecnologias de \textbf{Inteligência Artificial (IA)}, em particular o \textit{Deep Learning}, tem permitido uma penetração cada vez maior em setores tradicionalmente caracterizados por processos manuais e intensivos em conhecimento especializado, como a indústria de materiais \cite{cheng2022}. Este cenário tem impulsionado a aplicação de IA para automatizar tarefas complexas de inspeção e classificação visual.

O reconhecimento de padrões visuais, que antes dependia exclusivamente da experiência e acuidade humana, pode agora ser escalonado por meio de algoritmos de Visão Computacional. A aplicação dessas técnicas em materiais naturais, como rochas ornamentais e agregados, justifica-se pela complexidade e alta variabilidade dos padrões geológicos. A identificação precisa de um material envolve a análise de texturas, cores, veios e inclusões, elementos cruciais para determinar sua aplicação estética e funcional, bem como seu valor de mercado.

O sistema proposto neste trabalho busca desenvolver uma aplicação móvel inovadora, utilizando \textbf{redes neurais} para reconhecer e nomear diferentes tipos de rochas (como granito, mármore e quartzito) a partir de fotografias de superfícies instaladas, como pias, pisos e paredes. Esta iniciativa visa introduzir objetividade e padronização, impulsionando a eficiência econômica do setor e democratizando o acesso à informação técnica, auxiliando tanto leigos quanto profissionais na tomada de decisões.

\textcolor{red}{\textbf{Instrução Bolsista de INFORMÁTICA:} Acrescentar uma breve discussão sobre a evolução das abordagens de Visão Computacional, mencionando como as CNNs superaram as técnicas tradicionais (baseadas em SIFT, HOG) na análise de texturas complexas como as geológicas. Mencionar a necessidade de portabilidade (aplicativo).}

\section{Contexto e Desafios da Classificação de Materiais Pétreos}
A digitalização e a aplicação de tecnologias da informação têm se tornado imperativas no setor de construção e mineração, visando superar desafios de padronização e qualidade \cite{kumar2023}. O setor de materiais naturais, particularmente o segmento de rochas ornamentais, possui uma importância econômica global significativa, sendo um componente fundamental das indústrias de construção civil e design de interiores. O país em foco, sendo um dos principais produtores e exportadores mundiais, sublinha a relevância de otimizar os processos de classificação desses recursos.

Apesar da maturidade industrial, o setor enfrenta desafios práticos no que tange à identificação. As rochas ornamentais apresentam \textbf{alta variabilidade visual} dentro de uma mesma classificação comercial e diferenças sutis entre materiais distintos, dependendo da mineralogia, tamanho e distribuição dos grãos, e presença de estruturas secundárias. Para o consumidor e profissionais menos especializados, a identificação correta de uma laje a partir de uma amostra ou foto é uma tarefa complexa e propensa a erros.

As dificuldades práticas incluem a dependência de condições de iluminação, a avaliação subjetiva de textura e a interpretação de características microgeológicas visíveis. A ausência de ferramentas acessíveis para a identificação de rochas pode levar a escolhas inadequadas de materiais, impactando negativamente a estética e funcionalidade de projetos. Erros na escolha de materiais estão, inclusive, entre os fatores principais de retrabalhos em construções residenciais, conforme dados da indústria. Portanto, o desenvolvimento de um método de identificação rápido e objetivo, suportado por tecnologia, é crucial para mitigar esses riscos e aumentar a confiabilidade.

\textcolor{red}{\textbf{Instrução Bolsista de MINERAÇÃO (Análise Bibliográfica):} Realizar uma busca em bases de dados (Scopus, Google Scholar) sobre trabalhos recentes que aplicam IA/Visão Computacional na classificação de rochas ornamentais e/ou materiais geológicos. Identificar 2 a 3 artigos principais na área e, com base neles, escrever um parágrafo que descreva o \textbf{Estado da Arte} (o que já existe) e, em seguida, um parágrafo que defina claramente as \textbf{lacunas} que nosso trabalho preenche e as suas \textbf{principais contribuições} em relação à literatura.}

\textcolor{red}{\textbf{Instrução Bolsista de MINERAÇÃO:} Inserir uma citação relevante (e neutra, sem revelar autoria) sobre a importância da padronização e qualidade na classificação de rochas ornamentais (conforme o formato \cite{...}). Descrever mais detalhadamente 2 ou 3 dos principais fatores geológicos que tornam a diferenciação visual desafiadora (e.g., mudança de cor devido à oxidação, presença de veios ou fraturas).}

\section{Abordagens Computacionais para Reconhecimento de Padrões Visuais}
A utilização de \textit{Transfer Learning} em tarefas de classificação de imagens tem se consolidado como uma técnica poderosa para problemas com conjuntos de dados especializados e limitados \cite{garg2021}. Em geociências, onde a obtenção de grandes volumes de dados rotulados é custosa, essa abordagem é fundamental.

O reconhecimento de padrões visuais em larga escala foi transformado pelo advento das \textbf{Redes Neurais Convolucionais (CNNs)}. Essas arquiteturas são capazes de aprender hierarquicamente características complexas diretamente a partir dos dados de pixel, superando abordagens tradicionais baseadas em extração manual de \textit{features}. O \textit{transfer learning} consiste em utilizar pesos pré-treinados de modelos (como VGG, Inception ou ResNet) em grandes conjuntos de dados genéricos e ajustá-los (\textit{fine-tuning}) para uma tarefa específica.

Arquiteturas modernas como a \textbf{ResNet} (Residual Network), utilizada neste trabalho, são notáveis por introduzir as conexões residuais (\textit{skip connections}), que permitem o treinamento de redes muito profundas ao mitigar o problema do desvanecimento do gradiente (\textit{vanishing gradient}). Por sua vez, a família de arquiteturas \textbf{MobileNet} demonstra a possibilidade de realizar reconhecimento de padrões com alta performance e baixo custo computacional, o que é ideal para a eventual integração em aplicações móveis. A capacidade dessas redes em extrair e diferenciar texturas complexas e repetitivas as torna candidatas ideais para a análise de padrões em geociências, onde as características morfológicas e texturais são a base da classificação.

\textcolor{red}{\textbf{Instrução (Bolsista de INFORMÁTICA):} 
Justificar a escolha da arquitetura ResNet18, destacando o bom equilíbrio entre desempenho e simplicidade, ideal para esta fase inicial do projeto. 
Incluir também uma breve explicação sobre o uso de \textit{Data Augmentation}, isto é, técnicas que geram variações nas imagens originais (como rotações, espelhamentos e ajustes de brilho) para aumentar a diversidade do conjunto de treino e melhorar a capacidade de generalização do modelo.}


\section{Metodologia Proposta}

A metodologia adotada neste trabalho segue um fluxo de desenvolvimento iterativo e multidisciplinar, conforme ilustrado na Figura \ref{fig:metodologia}. O processo abrange desde a aquisição e o pré-processamento rigoroso dos dados geológicos (Seção 4.2) até o treinamento e a avaliação do modelo de \textit{Deep Learning} (Seção 4.3), culminando na validação de um modelo de prova de conceito para aplicação em contexto industrial.

\begin{figure}[ht]
\centering
\includegraphics[width=1\textwidth]{1.png}
\caption{Fluxo Metodológico.}
\label{fig:metodologia}
\end{figure}

\subsection{Escopo do Projeto e Classes de Materiais}
O escopo do projeto de desenvolvimento da aplicação móvel visa o completo ciclo de vida de um produto baseado em IA, desde a coleta e anotação dos dados até a implementação e validação em campo. Embora o desenvolvimento final da aplicação contemple diversas classes de rochas, nesta etapa de prova de conceito, a metodologia está concentrada na classificação e treinamento do modelo para as seguintes classes de materiais. A seleção destas classes foi determinada em função de um critério de \textbf{relevância industrial imediata}, alinhado com o fluxo de produção e as demandas específicas da empresa parceira de extensão tecnológica. Este foco estratégico permite que a prova de conceito seja validada em um cenário real e limitado, abordando materiais que apresentam similaridade visual desafiadora:
\begin{itemize}
    \item Granito Branco Itaúnas;
    \item Quartizito Perla;
    \item Quartizito Wakanda;
    \item Quartzito Verde Gaya.
\end{itemize}
O êxito na diferenciação dessas quatro classes visa validar a abordagem de \textit{Deep Learning} para a complexidade visual específica do problema.

\subsection{Coleta e Preparação de Imagens}
% --- Esta parte deve ser escrita pela bolsista da área de Mineração ---

A etapa de coleta de imagens foi planejada para garantir diversidade e representatividade adequada do conjunto de dados. Foram capturadas, aproximadamente, \textbf{150 imagens para cada uma das classes} de materiais listadas acima.
Cada chapa de material foi registrada por meio de aproximadamente 30 fotografias, distribuídas em diferentes distâncias, ângulos e condições de iluminação, utilizando dispositivos móveis.
O protocolo de captura foi definido da seguinte forma:

\begin{itemize}
    \item \textbf{10 imagens em regiões próximas}, destacando detalhes de textura e variações de padrão;
    \item \textbf{8 imagens em distância média}, capturando porções mais amplas da superfície;
    \item \textbf{5 imagens em ângulos inclinados}, buscando registrar efeitos de brilho e profundidade;
    \item \item \textbf{1 a 2 imagens da chapa inteira}, fornecendo visão geral do material;
    \item \textbf{3 a 5 imagens com iluminação ambiente diferenciada}, simulando condições reais de observação.
\end{itemize}

Esse protocolo contribui para reduzir viés de captura e ampliar a robustez do modelo de aprendizado. O conjunto de dados final foi particionado em 70\% para treino, 15\% para validação e 15\% para teste.

\textcolor{red}{\textbf{Instrução Bolsista de MINERAÇÃO (Para Versão Final):} Inserir uma figura (com subfiguras A, B, C, D) nesta subseção, mostrando um exemplo representativo de cada uma das 4 classes de rochas (Granito Branco Itaúnas, Quartizito Perla, Quartizito Wakanda, Quartzito Verde Gaya), com legenda clara.}

\begin{figure}[ht]
\centering
\includegraphics[width=1\textwidth]{2.png}
\caption{Figura 2}
\label{fig:1}
\end{figure}

\subsection{Treinamento do Modelo de Reconhecimento}
% --- Esta parte deve ser escrita pela bolsista da área de Informática ---

Para a etapa de reconhecimento de padrões, foi adotada uma abordagem baseada em \textit{transfer learning} com a arquitetura ResNet18. O modelo foi inicializado com pesos pré-treinados em ImageNet e ajustado para o conjunto de classes do problema em questão. A estratégia incluiu o congelamento das camadas convolucionais iniciais e a substituição da camada final por uma combinação de \textit{dropout} e camada totalmente conectada, específica para a classificação desejada.

O conjunto de dados foi particionado em três subconjuntos mutuamente exclusivos: treino (70\%), validação (15\%) e teste (15\%). O conjunto de treino foi utilizado para o ajuste iterativo dos pesos do modelo, enquanto o conjunto de validação monitorou a convergência e evitou o sobreajuste (\textit{overfitting}) durante as 15 épocas. O \textbf{conjunto de teste} foi reservado exclusivamente para a avaliação final e imparcial do desempenho do modelo treinado, garantindo a capacidade de generalização para dados não vistos.

Os hiperparâmetros foram configurados conforme segue: 15 épocas de treinamento, \textit{batch size} de 16, taxa de aprendizado inicial de $0.001$, otimizador Adam e função de perda \textit{Cross-Entropy Loss}. Um agendador de taxa de aprendizado do tipo \textit{StepLR} foi utilizado para reduzir a taxa em um fator de $0.1$ a cada 7 épocas, promovendo melhor convergência.

O pré-processamento das imagens incluiu redimensionamento, recortes aleatórios, rotações, inversões horizontais e verticais, além de ajustes sutis de brilho, contraste e saturação. Essas transformações aumentaram a variabilidade do conjunto de treino e reduziram o risco de sobreajuste (\textit{overfitting}).

Durante o treinamento, os melhores pesos foram salvos com base no maior valor de acurácia em validação, critério que assegurou a capacidade de generalização para dados não vistos.

A Tabela~\ref{tab:params} resume os principais parâmetros empregados no processo.

\begin{table}[h]
\centering
\begin{tabular}{|l|c|}
\hline
\textbf{Parâmetro} & \textbf{Valor adotado} \\ \hline
Épocas & 15 \\ \hline
Tamanho do lote (\textit{batch size}) & 16 \\ \hline
Taxa de aprendizado inicial & $0.001$ \\ \hline
Otimizador & Adam \\ \hline
Função de perda & Cross-Entropy Loss \\ \hline
Agendador de taxa & StepLR (fator $0.1$ a cada $7$ épocas) \\ \hline
Arquitetura & ResNet18 (camadas convolucionais congeladas) \\ \hline
Camada final & Dropout $(0.5)$ + Linear \\ \hline
\end{tabular}
\caption{Resumo dos principais parâmetros do treinamento da rede neural.}
\label{tab:params}
\end{table}

Os hiperparâmetros apresentados foram definidos a partir de testes exploratórios e de valores de referência comumente adotados em arquiteturas ResNet de pequeno porte. 
Não foi realizada uma busca sistemática de hiperparâmetros (grid search, random search ou métodos semelhantes), uma vez que esta etapa teve como principal objetivo avaliar a viabilidade técnica da abordagem proposta, e não a otimização de desempenho do modelo. 
A configuração final priorizou um equilíbrio entre acurácia adequada e custo computacional reduzido, considerando o caráter experimental e interdisciplinar do estudo.

\textcolor{red}{\textbf{Instrução (Bolsista de INFORMÁTICA):} 
Criar uma subseção (por exemplo, 4.3.1) descrevendo o ambiente de desenvolvimento e as ferramentas utilizadas (como Python, PyTorch ou TensorFlow, além das especificações do hardware e GPU disponíveis). Refazer também a descrição do treinamento (Seção 4.3) incluindo um método de busca sistemática de hiperparâmetros (por exemplo, \textit{Grid Search} ou \textit{Random Search}). Apresentar os intervalos de valores testados para cada parâmetro relevante (como taxa de aprendizado, tamanho do lote e número de épocas), o método de busca adotado e os valores finais selecionados com base na melhor acurácia de validação. O texto deve manter linguagem científica e explicitar o critério de escolha dos hiperparâmetros de forma reprodutível. \textbf{Ajuste no Script:} Garanta que o script de treinamento seja refatorado para realizar a separação estrita do conjunto de teste (15\%) e que a avaliação final do modelo seja feita utilizando este conjunto de teste reservado.}


\section{Resultados Esperados e Discussão}

A relevância científica deste trabalho reside na validação empírica de que arquiteturas de \textit{Deep Learning}, ajustadas via \textit{transfer learning}, podem alcançar alta performance na tarefa de discriminação de padrões visuais complexos presentes em materiais geológicos. Isso representa um avanço na aplicação da IA em setores de \textbf{baixa digitalização}.

Do ponto de vista social, o resultado mais significativo é a \textbf{democratização da informação}. Ao oferecer uma ferramenta acessível para a identificação confiável de materiais, o projeto apoia tanto o consumidor final quanto o profissional, que ganha um suporte técnico sem a necessidade de dispendiosa consultoria. O aplicativo em desenvolvimento integrará o modelo de IA a uma interface amigável. As telas principais da aplicação incluem:

\begin{enumerate}
    \item Uma interface de navegação que exibe a lista de rochas, recomendações e histórico de escaneamento.
    \item A interface principal de escaneamento, que permite a captura de imagens de rochas em diferentes ambientes.
    \item Uma tela de confirmação de reconhecimento, que exibe o nome da rocha e suas características geológicas relevantes.
\end{enumerate}

É importante reconhecer as \textbf{limitações e riscos}. O desempenho do modelo é intrinsecamente ligado à qualidade do conjunto de dados de treino, exigindo \textbf{atualização e retreinamento contínuos} frente à variação natural dos materiais. O risco de \textbf{dependência excessiva} do algoritmo deve ser mitigado pelo design da aplicação, que deve posicionar o sistema como uma ferramenta de apoio e não como um substituto integral do especialista.

A grande provocação é: em que medida a IA pode transformar as práticas tradicionais de identificação em geociências? O sistema proposto muda o paradigma da classificação, passando de um processo subjetivo, lento e centralizado no especialista, para um processo objetivo, instantâneo e escalável.

\textcolor{red}{\textbf{Instrução Bolsista de INFORMÁTICA (Para Versão Final):} Inserir os resultados quantitativos de desempenho do modelo nesta seção. Isso deve incluir: 1) Uma tabela ou gráfico com métricas como \textbf{Acurácia, Precisão, Recall e F1-Score} para o conjunto de teste; 2) Uma \textbf{Figura} mostrando o protótipo da interface do aplicativo (em smartphone ou web) com exemplos de uso (explicando a utilização e o fluxo do usuário), conforme a descrição das 3 telas principais.}

\begin{figure}[ht]
\centering
\includegraphics[width=1\textwidth]{3.png}
\caption{Telas principais do aplicativo.}
\label{fig:telas}
\end{figure}

\section{Considerações Finais}
O presente estudo explorou a interseção entre \textit{Deep Learning} e a indústria de materiais naturais, propondo uma metodologia robusta para o reconhecimento automatizado de padrões visuais em rochas. Retomando os pontos centrais, demonstramos a viabilidade do \textit{transfer learning} com a ResNet18 para lidar com a alta complexidade visual inerente aos materiais geológicos.

No contexto da integração e das \textbf{implicações futuras} do sistema, a escolha de uma arquitetura leve como a ResNet18 facilita seu empacotamento para uso em dispositivos de ponta (\textit{edge computing}), viabilizando a portabilidade e o uso em campo. Contudo, a implementação de modelos de IA exige considerações éticas, como a gestão de \textbf{vieses} algorítmicos e a necessidade de transparência. É crucial que o sistema indique o nível de confiança na classificação e reforce a validação humana em casos de baixa probabilidade.

\textcolor{red}{\textbf{Instrução Bolsista de INFORMÁTICA (Para Versão Final):} Elaborar um parágrafo detalhado sobre a manutenção e o ciclo de vida do modelo, abordando a necessidade de \textit{monitoring} contínuo do desempenho em campo e a estratégia de \textit{re-training} para garantir a precisão contínua frente à variação natural dos materiais e novas classificações. Este parágrafo deve ser inserido no bloco de Implicações na Conclusão.}

Enfatizamos a contribuição deste trabalho como uma \textbf{reflexão e provocação} dirigida à comunidade de Sistemas de Informação. Nosso foco é sublinhar o potencial da IA como agente de transformação e padronização em setores tradicionais. O sistema de reconhecimento aqui descrito é apresentado como uma \textbf{ideia emergente em desenvolvimento}, cujo objetivo é testar os limites da acurácia e generalização dos modelos, pavimentando o caminho para a criação de ferramentas tecnológicas acessíveis para o uso em campo.

\textcolor{red}{\textbf{Instrução CONJUNTA:} Assegurar que a \textbf{Bibliografia} (referências) contenha pelo menos 5 artigos/livros/fontes relevantes e neutras, conforme as orientações (IA, Visão Computacional, Setor de Rochas Ornamentais). Realizar a última checagem de anonimato.}

\bibliographystyle{sbc}
\bibliography{referencias}
% Sugestão de entradas BibTeX para o seu arquivo 'referencias.bib' (a serem verificadas e incluídas no arquivo pelo usuário):
% 
% @article{cheng2022,
%   title={{Deep learning applications in material science: a review}},
%   author={Cheng, L. and Zhang, H. and Wang, Q.},
%   journal={Journal of Materials Research and Technology},
%   volume={21},
%   pages={1332--1350},
%   year={2022},
%   publisher={Elsevier}
% }
% 
% @inproceedings{kumar2023,
%   title={{Digital transformation in the construction industry: A review of AI applications}},
%   author={Kumar, S. and Jain, P. and Singh, V.},
%   booktitle={Proceedings of the 2023 International Conference on Artificial Intelligence and Smart Systems (ICAIS)},
%   pages={1201--1207},
%   year={2023},
%   organization={IEEE}
% }
% 
% @article{garg2021,
%   title={{A comprehensive review of transfer learning for image classification}},
%   author={Garg, S. and Dhiman, G.},
%   journal={Artificial Intelligence Review},
%   volume={54},
%   issue={1},
%   pages={555--605},
%   year={2021},
%   publisher={Springer}
% }
% 
% (Adicionar mais referências neutras, como ABIROCHAS/FINDES/INBUILT, formatadas em BibTeX se forem usadas no texto).

\end{document}
