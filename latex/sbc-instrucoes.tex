\documentclass[12pt]{article}
\usepackage{sbc-template}
\usepackage{graphicx,url}
\usepackage[utf8]{inputenc}
\usepackage[brazil]{babel}
\usepackage{booktabs}
\usepackage{float}

\sloppy

\title{Classificação Automatizada de Rochas Ornamentais por Redes Neurais Convolucionais}

\author{Omitido para Revisão Cega}

\address{Omitido para Revisão Cega}

\begin{document}

\maketitle

\begin{abstract}
Este trabalho propõe um sistema de Visão Computacional para classificação automatizada de rochas ornamentais a partir de imagens digitais. Utilizando \textit{transfer learning} com a arquitetura ResNet18 e validação cruzada estratificada com 5 \textit{folds}, o modelo foi treinado para identificar cinco classes de materiais pétreos. A metodologia empregou aumento de dados e busca em grade para otimização de hiperparâmetros, alcançando acurácia média de 98,62\% (±0,83\%) e coeficiente Kappa de 0,9826. Os resultados indicam a viabilidade da aplicação de Redes Neurais Convolucionais para discriminação de padrões visuais em geociências, contribuindo para a transformação digital do setor de rochas ornamentais.
\end{abstract}

\section{Introdução}

O avanço das tecnologias de Inteligência Artificial (IA), em particular o \textit{Deep Learning}, tem permitido sua aplicação em setores caracterizados por processos manuais e intensivos em conhecimento especializado, como a indústria de rochas ornamentais \cite{cheng2022}. O reconhecimento de padrões visuais, que antes dependia da experiência humana, pode agora ser automatizado por algoritmos de Visão Computacional. Este cenário tem impulsionado a adoção de ferramentas baseadas em IA para automatizar tarefas de inspeção e classificação visual em diversos domínios industriais.

A identificação de rochas ornamentais envolve a análise de texturas, cores, veios e inclusões, elementos que determinam sua aplicação estética e funcional, bem como seu valor comercial. Para profissionais não especializados, essa identificação a partir de fotografias é uma tarefa complexa e propensa a erros. A ausência de ferramentas acessíveis pode levar a escolhas inadequadas de materiais, impactando projetos de construção e design. Erros na especificação de materiais são uma das causas frequentes de retrabalho em obras residenciais e comerciais.

A ruptura do paradigma de análise visual manual ocorreu com a consolidação das Redes Neurais Convolucionais (CNNs), que permitem o aprendizado de características diretamente a partir dos dados \cite{lecun2015deep}. A capacidade de realizar extração automática de atributos elimina a necessidade de engenharia manual de características, processo que tradicionalmente exigia conhecimento especializado do domínio \cite{soares2025}. A evolução das arquiteturas de \textit{Deep Learning} viabilizou também a execução desses modelos em dispositivos móveis, requisito relevante para inspeções em campo, onde o acesso a estações de trabalho dedicadas pode ser inviável.

O objetivo deste trabalho é desenvolver e validar um modelo de classificação de rochas ornamentais baseado em \textit{transfer learning}, avaliando sua capacidade de discriminar cinco classes de materiais com características visuais similares. Adicionalmente, propõe-se a integração deste modelo em uma aplicação móvel que permita a identificação de rochas a partir de fotografias capturadas em campo.

\section{Fundamentação Teórica}

A digitalização tem se tornado estratégica no setor de construção e mineração \cite{kumar2023}. O segmento de rochas ornamentais possui importância econômica significativa nas indústrias de construção civil e design de interiores. O Brasil ocupa posição de destaque no cenário mundial, figurando entre os principais produtores e exportadores de granitos, mármores e quartzitos.

Apesar da maturidade industrial, o setor enfrenta desafios práticos de identificação. As rochas ornamentais apresentam variabilidade visual dentro de uma mesma classificação comercial e diferenças sutis entre materiais distintos, dependendo da mineralogia, tamanho e distribuição dos grãos, bem como da presença de estruturas secundárias. As dificuldades incluem a dependência de condições de iluminação, a avaliação subjetiva de textura e a interpretação de características microgeológicas visíveis a olho nu. A nomenclatura comercial frequentemente diverge da classificação geológica, gerando confusões que podem resultar em especificações incorretas de materiais.

O uso de IA para reconhecimento de padrões em materiais naturais evoluiu de soluções baseadas em processamento de imagens tradicionais, como histogramas e análise de cores, para abordagens automatizadas baseadas em aprendizado profundo \cite{liu2019deep}. Métodos tradicionais como SIFT (\textit{Scale-Invariant Feature Transform}) e HOG (\textit{Histogram of Oriented Gradients}), embora eficazes para objetos com geometria rígida, apresentam limitações ao lidar com a complexidade das superfícies naturais. A utilização de \textit{transfer learning} consolidou-se como técnica eficaz para problemas com conjuntos de dados limitados \cite{garg2021}, permitindo aproveitar conhecimento aprendido em grandes bases de dados genéricas como o ImageNet.

A arquitetura ResNet (\textit{Residual Network}) introduziu as conexões residuais (\textit{skip connections}), que permitem o treinamento de redes profundas ao mitigar o problema do desvanecimento do gradiente \cite{he2016deep}. A ResNet18, com suas 18 camadas, oferece equilíbrio entre capacidade de representação e custo computacional \cite{Silva2021}. Essa característica torna a arquitetura adequada para aplicações em dispositivos com recursos limitados, como smartphones e tablets utilizados em inspeções de campo.

A técnica de aumento de dados (\textit{data augmentation}) consiste em aplicar transformações nas imagens de treino, como rotações, espelhamentos, recortes e ajustes de luminosidade, gerando variações que forçam a rede a aprender características invariantes às condições de captura \cite{shorten2019survey}. Esta abordagem reduz o risco de sobreajuste (\textit{overfitting}) e melhora a capacidade de generalização do modelo em cenários reais \cite{srivastava2014dropout}. A combinação de \textit{transfer learning} com aumento de dados tem se mostrado particularmente eficaz em domínios com escassez de dados rotulados.

\section{Metodologia}

A metodologia adotada segue um fluxo de desenvolvimento iterativo e multidisciplinar, conforme ilustrado na Figura~\ref{fig:metodologia}, abrangendo desde a aquisição e o pré-processamento dos dados geológicos até o treinamento e a avaliação do modelo de \textit{Deep Learning}.

\begin{figure}[H]
\centering
\includegraphics[width=0.9\textwidth]{fig_metodologia.png}
\caption{Fluxo metodológico do sistema proposto.}
\label{fig:metodologia}
\end{figure}

\subsection{Conjunto de Dados}

A etapa de coleta de imagens foi planejada para garantir diversidade e representatividade adequada do conjunto de dados. Foram capturadas aproximadamente 150 imagens para cada uma das cinco classes de materiais: Granito Branco Itaúnas, Mármore Matarazzo, Quartzito Perla, Quartzito Wakanda e Quartzito Verde Gaya. Cada material foi fotografado em diferentes distâncias, ângulos e condições de iluminação, utilizando dispositivos móveis.

Cabe ressaltar que o contexto industrial deste projeto contempla aproximadamente 20 classes diferentes de rochas ornamentais, as quais serão incorporadas progressivamente ao modelo até sua versão final. A seleção inicial de cinco classes deve-se ao fluxo de produção vigente, uma vez que a captura das imagens está condicionada à disponibilidade dos materiais durante o processo produtivo.

O protocolo de captura incluiu: imagens em regiões próximas para detalhes de textura; imagens em distância média para visão geral do padrão; imagens em ângulos inclinados para registrar efeitos de brilho e profundidade; e imagens com iluminação diferenciada (natural e artificial) para simular condições reais de observação. Este protocolo visa reduzir viés de captura e ampliar a robustez do modelo frente a variações encontradas em ambientes reais de uso.

A Figura~\ref{fig:amostras} apresenta exemplos das classes utilizadas, evidenciando a variabilidade de padrões visuais e texturas características de cada material. Observa-se que, apesar de pertencerem a categorias geológicas distintas, alguns materiais apresentam similaridades visuais que tornam a tarefa de classificação desafiadora para observadores não especializados.

\begin{figure}[H]
\centering
\includegraphics[width=0.95\textwidth]{fig_amostras_rochas.png}
\caption{Exemplos das classes de rochas: (a) Granito Branco Itaúnas, (b) Mármore Matarazzo, (c) Quartzito Perla, (d) Quartzito Wakanda, (e) Quartzito Verde Gaya.}
\label{fig:amostras}
\end{figure}

\subsection{Arquitetura e Treinamento}

Foi adotada a arquitetura ResNet18 com pesos pré-treinados no ImageNet, uma base de dados com mais de 14 milhões de imagens em 1.000 categorias. As camadas convolucionais foram congeladas para funcionar como extrator de características, aproveitando o conhecimento prévio da rede. A camada final foi substituída por camadas \textit{dropout} (com taxas de 50\% e 30\%) e camadas totalmente conectadas para classificação das cinco classes. Esta estratégia reduz o número de parâmetros treináveis para aproximadamente 68.000, diminuindo o risco de sobreajuste e acelerando o treinamento.

Para avaliação do modelo, foi empregada validação cruzada estratificada com $k=5$ partições. O pré-processamento incluiu técnicas de aumento de dados aplicadas apenas durante o treinamento: recortes aleatórios redimensionados, inversões horizontais e verticais, rotações de até 30°, ajustes de brilho, contraste e saturação, e desfoque gaussiano. É importante ressaltar que a avaliação foi realizada sobre as imagens originais, sem transformações de aumento, garantindo que as métricas reflitam o desempenho real do modelo.

A seleção de hiperparâmetros foi realizada por busca em grade (\textit{Grid Search}), avaliando combinações de taxa de aprendizado ($0.0001$, $0.0005$, $0.001$), tamanho de lote (8, 16, 32) e otimizador (Adam, SGD, AdamW), totalizando 27 combinações. Cada combinação foi submetida à validação cruzada completa, resultando em 135 execuções de treinamento independentes. A Tabela~\ref{tab:params} apresenta a configuração selecionada.

\begin{table}[H]
\centering
\caption{Parâmetros do treinamento.}
\label{tab:params}
\begin{tabular}{lc}
\toprule
\textbf{Parâmetro} & \textbf{Valor} \\
\midrule
Arquitetura & ResNet18 (pré-treinado) \\
Taxa de aprendizado & $0.001$ \\
Tamanho do lote & 16 \\
Otimizador & Adam \\
Função de perda & Cross-Entropy (pesos de classe) \\
Validação cruzada & $k=5$ \textit{folds} \\
\textit{Early stopping} & 7 épocas \\
Tamanho da imagem & $224 \times 224$ pixels \\
\bottomrule
\end{tabular}
\end{table}

\section{Resultados}

A validação cruzada estratificada obteve acurácia média de 98,62\% (±0,83\%). A baixa variância entre os \textit{folds} indica que o modelo apresenta desempenho consistente independentemente da partição utilizada para validação. A Tabela~\ref{tab:fold_results} apresenta os resultados por \textit{fold}.

\begin{table}[H]
\centering
\caption{Acurácia por \textit{fold}.}
\label{tab:fold_results}
\begin{tabular}{cc}
\toprule
\textbf{Fold} & \textbf{Acurácia} \\
\midrule
1 & 98,11\% \\
2 & 98,74\% \\
3 & 98,74\% \\
4 & 100,00\% \\
5 & 97,48\% \\
\midrule
\textbf{Média $\pm$ DP} & \textbf{98,62\% $\pm$ 0,83\%} \\
\bottomrule
\end{tabular}
\end{table}

A Tabela~\ref{tab:classification_metrics} apresenta precisão, revocação e F1-\textit{score} por classe. Destaca-se que o Granito Branco Itaúnas obteve 100\% de revocação (todas as amostras corretamente identificadas), enquanto o Quartzito Verde Gaya apresentou as métricas mais equilibradas.

\begin{table}[H]
\centering
\caption{Métricas por classe.}
\label{tab:classification_metrics}
\begin{tabular}{lcccc}
\toprule
\textbf{Classe} & \textbf{Precisão} & \textbf{Revocação} & \textbf{F1} & \textbf{Suporte} \\
\midrule
Granito Branco Itaúnas & 0,98 & 1,00 & 0,99 & 160 \\
Mármore Matarazzo & 0,99 & 0,97 & 0,98 & 156 \\
Quartzito Perla & 0,97 & 0,99 & 0,98 & 138 \\
Quartzito Wakanda & 0,99 & 0,98 & 0,99 & 134 \\
Quartzito Verde Gaya & 1,00 & 1,00 & 1,00 & 207 \\
\midrule
\textbf{Média Ponderada} & 0,99 & 0,99 & 0,99 & 795 \\
\bottomrule
\end{tabular}
\end{table}

O coeficiente Kappa de Cohen foi de 0,9826, indicando concordância quase perfeita entre as predições do modelo e as classes verdadeiras. Este valor supera o limiar de 0,81 tipicamente considerado como concordância forte na literatura.

\subsection{Análise de Convergência}

A Figura~\ref{fig:curvas_aprendizado} apresenta as curvas de aprendizado para os 5 \textit{folds} da validação cruzada. A análise revela convergência consistente: as curvas de perda decrescem de forma suave, estabilizando-se nas épocas finais. O paralelismo entre as curvas de treino e validação indica ausência de sobreajuste significativo. Os valores de acurácia de validação atingem patamares superiores a 95\% já nas primeiras épocas.

\begin{figure}[H]
\centering
\includegraphics[width=0.75\textwidth]{learning_curves.png}
\caption{Curvas de aprendizado: (superior) perda; (inferior) acurácia. Linhas tracejadas representam treinamento e linhas sólidas representam validação.}
\label{fig:curvas_aprendizado}
\end{figure}

A Figura~\ref{fig:matriz_confusao} apresenta a matriz de confusão agregada, evidenciando que a maioria das amostras foi classificada corretamente.

\begin{figure}[H]
\centering
\includegraphics[width=0.55\textwidth]{confusion_matrix.png}
\caption{Matriz de confusão agregada dos 5 \textit{folds}.}
\label{fig:matriz_confusao}
\end{figure}

\subsection{Discussão}

A acurácia obtida pode ser justificada por diversos fatores metodológicos e características do domínio: (1) a validação cruzada estratificada garante resultados não dependentes de uma divisão específica dos dados; (2) múltiplas técnicas de regularização foram empregadas, incluindo \textit{dropout}, \textit{early stopping} e aumento de dados; (3) o congelamento das camadas convolucionais reduziu o número de parâmetros treináveis para aproximadamente 68.000; (4) rochas ornamentais apresentam padrões visuais distintos (granitos com grãos cristalinos, mármores com veios característicos, quartzitos com padrões cromáticos únicos) que facilitam a discriminação por CNNs.

Contudo, limitações devem ser reconhecidas: o desempenho está associado à qualidade do conjunto de dados, e o modelo pode requerer retreinamento para novos materiais.

\section{Aplicação Prática}

Do ponto de vista prático, o sistema visa democratizar o acesso à informação técnica, apoiando consumidores e profissionais na identificação de materiais. O aplicativo móvel desenvolvido integra o modelo de classificação a uma interface amigável, permitindo a identificação de rochas ornamentais diretamente em campo. A Figura~\ref{fig:app_interface} apresenta a interface do aplicativo, ilustrando o fluxo de uso.

\begin{figure}[H]
\centering
\includegraphics[width=0.85\textwidth]{fig_app_interface.png}
\caption{Interface do aplicativo: (a) tela inicial com histórico de escaneamentos, (b) interface de captura de imagem, (c) resultado com identificação da rocha e nível de confiança.}
\label{fig:app_interface}
\end{figure}

O fluxo de uso consiste em: (1) acesso à tela inicial com histórico de classificações anteriores; (2) captura de imagem da rocha via câmera do dispositivo; (3) processamento pelo modelo de classificação; (4) exibição do resultado com identificação da rocha e nível de confiança da predição. A arquitetura leve da ResNet18 viabiliza a execução em dispositivos móveis sem necessidade de conexão com servidores externos.

Como trabalho futuro, planeja-se a expansão do modelo para contemplar as aproximadamente 20 classes de rochas ornamentais do contexto industrial, além da coleta contínua de novas amostras para enriquecimento do conjunto de dados e melhoria da capacidade de generalização.

\section{Considerações Finais}

A contribuição deste trabalho reside na validação de que arquiteturas de \textit{Deep Learning}, ajustadas via \textit{transfer learning}, podem discriminar padrões visuais complexos em materiais geológicos. Os resultados obtidos (acurácia de 98,62\% e Kappa de 0,9826) indicam a viabilidade técnica da abordagem para aplicações no setor de rochas ornamentais.

Trabalhos futuros incluem expansão do número de classes, coleta de dados em condições variadas, desenvolvimento do aplicativo móvel e implementação de ciclo MLOps.

\bibliographystyle{sbc}
\bibliography{referencias}

\end{document}
